\documentclass{article}

\usepackage[margin=1in]{geometry}

\newcommand{\figref}[1]{Figure~\ref{fig:#1}}

\begin{document}

\begin{titlepage}
\begin{center}

  \vspace*{1cm}

  \Huge
    \textbf{Parallel Pied Pipers}

  \vspace{0.5cm}
  \LARGE
    Programming and Problem Solving, Project 1

    \vspace{1.5cm}

  \textbf{Richard Townsend\\ Gil Feig\\ Guarav Ahuja}

  \vfill
    
    September 29, 2014
    \vspace{0.8cm}



\end{center}
\end{titlepage}


\tableofcontents

\newpage

\section{Introduction}
%Provide overview of this document (and the course?)

\section{Problem Description}
%Restate problem concisely and clearly

\section{Strategies}

\subsection{Zone Sweeping}

The first strategy we employed was divide-and-conquer in nature, and combined a few key insights
mentioned in class. In this 'zone sweeping' technique, we evenly divide the right side of the field into
$d$ rectangular zones and each piper is assigned to a zone by matching a piper's id with the location
of the zone (the top-most zone is assinged to piper 0, the lowest zone to piper $d-1$). (SEE FIGURE OF ZONES)
Every piper runs to the middle of their zone and positions himself 10m from the rightmost fence.
Upon reaching this location, a piper enters a 'collection' state and starts playing music. While in the
collection state, a piper takes the average of both the x- and y-coordinate of every rat in that piper's
zone. This calculation omits any rats already under that piper's spell. This average rat coordinate becomes
the target location for the piper to reach. We recalculate this target on every tick until all the rats
in a piper's zone are under his influence. Once a piper's zone is clear, he runs back to the gate and moves
10m into the left side of the field before releasing all the captured rats so far. The piper then returns to
the far right edge of his zone and repeats the process.

\subsection{Conveyor Belt}

\section{Analysis}

\section{Conclusions}
%Discuss future improvements that could be made.
%Remark on how this problem made us think.


\newpage

\bibliographystyle{abbrvnat}

% The bibliography should be embedded for final submission.
\bibliography{sedwards}

\end{document}
